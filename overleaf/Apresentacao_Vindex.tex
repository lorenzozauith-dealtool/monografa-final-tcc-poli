% Apresentação TCC - Projeto Vindex
% Template Beamer para Escola Politécnica da USP
\documentclass[aspectratio=169]{beamer}

% Tema e cores
\usetheme{Madrid}
\usecolortheme{default}
\setbeamercolor{structure}{fg=blue!70!black}
\setbeamercolor{title}{fg=blue!80!black}
\setbeamertemplate{navigation symbols}{}
\setbeamertemplate{footline}[frame number]

% Pacotes
\usepackage[brazil]{babel}
\usepackage[utf8]{inputenc}
\usepackage{graphicx}
\usepackage{booktabs}
\usepackage{amsmath}
\usepackage{tikz}
\usetikzlibrary{arrows,positioning}

% Informações do documento
\title[Vindex - Detecção de Acidentes]{Sistema de Detecção de Acidentes e Alerta Automático para Motociclistas}
\subtitle{Projeto Vindex}
\author[Lacerda, Zauith, Araújo]{Igor Teixeira Lacerda \\ Lorenzo Bighetti Zauith \\ Rodrigo Costa de Araújo}
\institute[Poli-USP]{Escola Politécnica da Universidade de São Paulo \\ Departamento de Engenharia Elétrica}
\date{Dezembro de 2025}

\begin{document}

% Slide 1: Título
\begin{frame}
\titlepage
\end{frame}

% Slide 2: Sumário
\begin{frame}{Sumário}
\tableofcontents
\end{frame}

% Seção 1: Introdução
\section{Introdução}

\begin{frame}{Contexto e Motivação}
\begin{block}{Problema}
\begin{itemize}
    \item A cada 39 minutos, um motociclista morre no trânsito brasileiro
    \item Taxa de mortalidade 30× maior que ocupantes de automóveis
    \item Tempo de resposta crítico: \textbf{"hora de ouro"} (60 minutos)
    \item 60\% mais chances de sobreviver se socorrido nos primeiros 30 minutos
\end{itemize}
\end{block}

\begin{block}{Motivação}
Reduzir o tempo entre o acidente e o atendimento à vítima através de \textbf{detecção automática} e \textbf{alerta imediato}.
\end{block}
\end{frame}

\begin{frame}{Declaração do Problema}
\begin{alertblock}{Questão Central}
\textbf{Como detectar automaticamente um acidente envolvendo uma motocicleta e notificar, de forma confiável e imediata, os contatos de emergência ou serviços de resgate?}
\end{alertblock}

\vspace{0.5cm}

\textbf{Subproblemas técnicos:}
\begin{itemize}
    \item Distinguir acidente real de eventos cotidianos (buracos, frenagens)
    \item Garantir que o alerta chegue rapidamente mesmo em condições adversas
    \item Fornecer informações adicionais para melhorar eficiência do socorro
\end{itemize}
\end{frame}

% Seção 2: Objetivos
\section{Objetivos}

\begin{frame}{Objetivos do Projeto}
\begin{block}{Objetivo Geral}
Desenvolver um sistema embarcado de baixo custo para \textbf{detecção automática de acidentes motociclísticos} com alerta de emergência em tempo real.
\end{block}

\vspace{0.3cm}

\begin{block}{Objetivos Específicos}
\begin{enumerate}
    \item \textbf{Detectar acidentes} usando sensores inerciais (acelerômetro + giroscópio)
    \item \textbf{Transmitir alertas} via Bluetooth e internet para contatos de emergência
    \item \textbf{Registrar evidências} visuais com câmera e visão computacional (YOLO + OCR)
\end{enumerate}
\end{block}
\end{frame}

\begin{frame}{Requisitos Funcionais}
\begin{table}
\centering
\small
\begin{tabular}{cl}
\toprule
\textbf{RF} & \textbf{Descrição} \\
\midrule
RF-01 & Detecção de evento crítico (aceleração + rotação) \\
RF-02 & Acionamento do módulo de captura \\
RF-03 & Captura de imagens/vídeo (8 segundos) \\
RF-04 & Transmissão ao smartphone via BLE \\
RF-05 & Processamento no app (YOLO + OCR) \\
RF-06 & Envio de alerta automático \\
RF-07 & Interação com usuário (cancelamento) \\
RF-08 & Registro de telemetria contínua \\
\bottomrule
\end{tabular}
\end{table}
\end{frame}

\begin{frame}{Requisitos Não-Funcionais}
\begin{columns}[T]
\begin{column}{0.48\textwidth}
\textbf{Desempenho:}
\begin{itemize}
    \item Autonomia: 10 horas
    \item Tempo de resposta: < 30 segundos
    \item Sensibilidade: > 95\%
    \item Especificidade: < 5\% falsos positivos
\end{itemize}
\end{column}

\begin{column}{0.48\textwidth}
\textbf{Qualidade:}
\begin{itemize}
    \item Robustez física (IP54)
    \item Usabilidade intuitiva
    \item Conectividade estável (BLE 5m)
    \item Custo: R\$ 300-400
\end{itemize}
\end{column}
\end{columns}
\end{frame}

% Seção 3: Materiais e Métodos
\section{Materiais e Métodos}

\begin{frame}{Arquitetura do Sistema}
\begin{block}{Três Camadas Principais}
\begin{enumerate}
    \item \textbf{Hardware Embarcado} (motocicleta)
    \begin{itemize}
        \item Node MCU: monitoramento contínuo, detecção de acidentes
        \item Placa Computacional: captura de vídeo, upload para nuvem
        \item Sensor BMI160: acelerômetro + giroscópio (IMU)
        \item Câmera OV5647: 160° FOV, 720p
    \end{itemize}
    
    \item \textbf{Aplicativo Móvel} (iOS)
    \begin{itemize}
        \item Conexão BLE contínua em background
        \item Processamento de imagem (YOLO + OCR)
        \item Envio de alertas ao backend
    \end{itemize}
    
    \item \textbf{Serviços em Nuvem} (Supabase + Telegram)
    \begin{itemize}
        \item Armazenamento de vídeos
        \item Bot Telegram para notificação do guardião
    \end{itemize}
\end{enumerate}
\end{block}
\end{frame}

\begin{frame}{Componentes de Hardware}
\begin{columns}[T]
\begin{column}{0.48\textwidth}
\textbf{Node MCU:}
\begin{itemize}
    \item RISC-V 32-bit, 160 MHz
    \item BLE 5.0 integrado
    \item Consumo: 5 µA (deep sleep)
    \item Dimensões: 22,52 × 18 mm
\end{itemize}

\vspace{0.3cm}

\textbf{Sensor BMI160:}
\begin{itemize}
    \item Acelerômetro: ±16g
    \item Giroscópio: ±2000°/s
    \item Taxa: 100 Hz
\end{itemize}
\end{column}

\begin{column}{0.48\textwidth}
\textbf{Placa Computacional:}
\begin{itemize}
    \item ARM Cortex-A53 1.0 GHz
    \item 512 MB RAM
    \item Wi-Fi + Bluetooth 4.2
    \item Consumo: 200 mA (idle)
\end{itemize}

\vspace{0.3cm}

\textbf{Câmera OV5647:}
\begin{itemize}
    \item 5 MP, 160° FOV
    \item 720p @ 30 fps
    \item Buffer circular de 8s
\end{itemize}
\end{column}
\end{columns}
\end{frame}

\begin{frame}{Evolução do Projeto}
\textbf{Processo iterativo com 5 fases:}

\begin{enumerate}
    \item \textbf{Fase I:} Tentativa de integração com Telegram nativo
    \begin{itemize}
        \item Fracasso: incompatibilidades iOS, restrições de privacidade
    \end{itemize}
    
    \item \textbf{Fase II:} Solução via Bot Telegram + Supabase
    \begin{itemize}
        \item Sucesso: contornou restrições iOS
    \end{itemize}
    
    \item \textbf{Fase III:} Tentativa com Node MCU-CAM
    \begin{itemize}
        \item Limitação: imagens estáticas insuficientes
    \end{itemize}
    
    \item \textbf{Fase IV:} Experimentação com Node MCU-S3
    \begin{itemize}
        \item Limitação: memória insuficiente para vídeo de qualidade
    \end{itemize}
    
    \item \textbf{Fase V:} Arquitetura final com upload para nuvem
    \begin{itemize}
        \item Solução: Placa Computacional + upload via Wi-Fi
    \end{itemize}
\end{enumerate}
\end{frame}

\begin{frame}{Algoritmo de Detecção}
\begin{block}{Critérios de Detecção}
\textbf{Acidente detectado quando:}
\begin{itemize}
    \item Aceleração resultante $|\vec{a}| > 3,5g$ \textbf{OU}
    \item Inclinação superior a 65° (variação angular abrupta)
\end{itemize}
\end{block}

\vspace{0.3cm}

\textbf{Fluxo de detecção:}
\begin{enumerate}
    \item Leitura contínua do sensor BMI160 a 100 Hz
    \item Filtragem para separar gravidade de aceleração dinâmica
    \item Verificação de critérios (aceleração + rotação)
    \item Timer de confirmação (100 ms)
    \item Protocolo de alerta ativado
\end{enumerate}
\end{frame}

\begin{frame}{Fluxo de Alerta}
\begin{enumerate}
    \item \textbf{Detecção:} Node MCU detecta acidente (0,5s)
    \item \textbf{Despertar:} Node MCU acorda Placa Computacional via GPIO
    \item \textbf{Captura:} Placa Computacional salva vídeo de 8s (buffer circular)
    \item \textbf{Notificação BLE:} Node MCU envia alerta ao smartphone
    \item \textbf{Confirmação:} App exibe alerta sonoro/visual (15s para cancelar)
    \item \textbf{Localização:} App obtém GPS atual
    \item \textbf{Processamento:} YOLO detecta placa, OCR extrai caracteres
    \item \textbf{Upload:} Vídeo enviado para Supabase (5s)
    \item \textbf{Alerta:} Bot Telegram notifica guardião (4,5s)
\end{enumerate}

\vspace{0.3cm}
\textbf{Tempo total médio: 28 segundos}
\end{frame}

\begin{frame}{Design do Case}
\begin{block}{Características do Invólucro}
\textbf{Projeto específico para o hardware:}
\begin{itemize}
    \item Ferramentas: TinkerCad + Inventor (AutoCAD)
    \item Material: PLA (resistência, leveza, baixo custo)
    \item Proteção: IP54 (poeira e respingos)
    \item Vedação: anel de silicone nas junções
\end{itemize}
\end{block}

\vspace{0.3cm}

\textbf{Três furos estratégicos:}
\begin{enumerate}
    \item Região frontal: câmera
    \item Lateral: botão de switch
    \item Traseira: carregamento USB
\end{enumerate}
\end{frame}

% Seção 4: Resultados
\section{Resultados}

\begin{frame}{Desempenho da Detecção}
\begin{table}
\centering
\begin{tabular}{lc}
\toprule
\textbf{Métrica} & \textbf{Resultado} \\
\midrule
Cenários simulados & 10 acidentes \\
Detecções corretas & 9 de 10 (90\%) \\
Falsos positivos & 0 de 12h pilotagem (0\%) \\
Sensibilidade & \textbf{90\%} \\
Especificidade & \textbf{100\%} \\
\midrule
Tempo de resposta médio & \textbf{28 segundos} \\
Autonomia & \textbf{10 horas} \\
\bottomrule
\end{tabular}
\end{table}

\vspace{0.3cm}
\textbf{Conclusão:} Sistema atende requisitos RNF-03 e RNF-04
\end{frame}

\begin{frame}{Tempo de Resposta Detalhado}
\begin{table}
\centering
\begin{tabular}{lc}
\toprule
\textbf{Etapa} & \textbf{Tempo (s)} \\
\midrule
Detecção do acidente (Node MCU) & 0,5 \\
Confirmação (usuário ou timeout) & 15,0 \\
Processamento e envio no app & 3,0 \\
Transmissão na nuvem (trigger + bot) & 5,0 \\
Notificação no Telegram & 4,5 \\
\midrule
\textbf{Tempo total} & \textbf{28,0} \\
\bottomrule
\end{tabular}
\end{table}

\vspace{0.3cm}
\textbf{Meta RNF-04:} < 30 segundos ✓
\end{frame}

\begin{frame}{Qualidade das Evidências Visuais}
\begin{block}{Imagens/Fotos (BLE)}
\begin{itemize}
    \item Resolução: 640×480 (< 50 kB)
    \item Qualidade: suficiente para identificar veículos/cenário
    \item Limitação: 2 casos noturnos com imagem escura (corrigido via software)
\end{itemize}
\end{block}

\begin{block}{Vídeos (720p, 8s)}
\begin{itemize}
    \item Revelam dinâmica exata da queda e objetos ao redor
    \item Tamanho: ~3,5 MB
    \item Upload: 5 segundos em 4G
\end{itemize}
\end{block}

\begin{block}{Detecção de Placa (YOLO + OCR)}
\begin{itemize}
    \item Acurácia: \textbf{95\%} em condições favoráveis
    \item 50 imagens testadas: 47 placas identificadas corretamente
    \item OCR: 94\% de acerto após correções (O/0, I/1)
\end{itemize}
\end{block}
\end{frame}

\begin{frame}{Testes Realizados}
\begin{columns}[T]
\begin{column}{0.48\textwidth}
\textbf{Testes de Detecção:}
\begin{itemize}
    \item Quedas bruscas: 5/5 detectadas
    \item Buracos (20 km/h): 0 falsos alarmes
    \item Frenagens (40 km/h): 0 falsos alarmes
    \item Colisões (15 km/h): 4/5 detectadas
\end{itemize}
\end{column}

\begin{column}{0.48\textwidth}
\textbf{Testes de Comunicação:}
\begin{itemize}
    \item Alcance BLE: 8m (sem obstáculos)
    \item Alcance BLE: 5m (com jaqueta)
    \item Reconexão: 3s após perda
    \item Upload vídeo: 5s (3,5 MB em 4G)
\end{itemize}
\end{column}
\end{columns}

\vspace{0.5cm}
\begin{alertblock}{Resultado}
\textbf{90\% de sensibilidade} e \textbf{100\% de especificidade} nos testes realizados
\end{alertblock}
\end{frame}

% Seção 5: Conclusão
\section{Conclusão}

\begin{frame}{Contribuições Principais}
\begin{enumerate}
    \item Sistema de detecção combinando \textbf{aceleração (3,5g) + inclinação (65°)}
    \begin{itemize}
        \item Identifica colisões frontais e tombamentos laterais
    \end{itemize}
    
    \item Arquitetura híbrida \textbf{Node MCU + Placa Computacional}
    \begin{itemize}
        \item Otimiza autonomia energética (10 horas)
    \end{itemize}
    
    \item Visão computacional \textbf{YOLO + OCR} para identificação de placas
    \begin{itemize}
        \item Funcionalidade ausente em sistemas comerciais de baixo custo
    \end{itemize}
    
    \item Validação em testes controlados
    \begin{itemize}
        \item 90\% detecção, 0\% falsos alarmes, 28s resposta
    \end{itemize}
\end{enumerate}
\end{frame}

\begin{frame}{Limitações Identificadas}
\begin{itemize}
    \item \textbf{Transferência de vídeo:} requer cobertura de rede (pode falhar em áreas remotas)
    
    \item \textbf{Resolução de imagens BLE:} limitada a 640×480 (qualidade reduzida para análise forense)
    
    \item \textbf{Algoritmo de limiares fixos:} pode não capturar acidentes de baixa energia cinética
    
    \item \textbf{Dependência de smartphone:} ponto de falha se celular descarregado ou fora de alcance BLE
\end{itemize}
\end{frame}

\begin{frame}{Trabalhos Futuros}
\begin{enumerate}
    \item \textbf{Aprendizado de máquina embarcado}
    \begin{itemize}
        \item Detecção mais robusta, adaptação a estilos de pilotagem
    \end{itemize}
    
    \item \textbf{Miniaturização em PCB dedicada}
    \begin{itemize}
        \item Redução de tamanho e consumo
    \end{itemize}
    
    \item \textbf{Node MCU-S3 com aceleração de IA}
    \begin{itemize}
        \item Eliminar dependência da Placa Computacional
    \end{itemize}
    
    \item \textbf{Versão Android do aplicativo}
    \begin{itemize}
        \item Ampliar base de usuários
    \end{itemize}
    
    \item \textbf{Integração direta com SAMU/Bombeiros}
    \begin{itemize}
        \item Aumentar eficácia do socorro
    \end{itemize}
    
    \item \textbf{Testes em larga escala}
    \begin{itemize}
        \item Coletar dados reais, refinar calibragem
    \end{itemize}
\end{enumerate}
\end{frame}

\begin{frame}{Impacto Social}
\begin{block}{Potencial de Salvar Vidas}
\begin{itemize}
    \item Redução significativa do tempo pré-hospitalar
    \item Dispositivo pode cortar \textbf{dezenas de minutos} do tempo de resposta
    \item Aumento substancial das chances de sobrevivência
\end{itemize}
\end{block}

\vspace{0.3cm}

\begin{block}{Benefícios Adicionais}
\begin{itemize}
    \item Registro de dados do acidente para análise
    \item Cultura de responsabilização e aprendizado
    \item Estatísticas mais precisas para autoridades
    \item Evidências para processos de seguros
\end{itemize}
\end{block}
\end{frame}

\begin{frame}{Conclusão Final}
\begin{block}{Viabilidade Técnica Demonstrada}
Este trabalho demonstrou a viabilidade técnica de um \textbf{dispositivo acessível} (R\$ 300-400) para detecção automática de acidentes motociclísticos.
\end{block}

\vspace{0.5cm}

\begin{alertblock}{Resultados Alcançados}
\begin{itemize}
    \item $\checkmark$ Detecção: 90\% de sensibilidade, 0\% falsos alarmes
    \item $\checkmark$ Tempo de resposta: 28 segundos (meta: < 30s)
    \item $\checkmark$ Autonomia: 10 horas
    \item $\checkmark$ Visão computacional: 95\% acurácia OCR
    \item $\checkmark$ Custo acessível: R\$ 300-400
\end{itemize}
\end{alertblock}

\vspace{0.3cm}
\textbf{Solução viável tecnicamente e pode contribuir para redução da mortalidade de motociclistas no trânsito brasileiro.}
\end{frame}

% Slide final: Perguntas
\begin{frame}[plain]
\begin{center}
\Huge{\textbf{Obrigado!}}

\vspace{1cm}

\Large{Perguntas?}

\vspace{1cm}

\normalsize
\textbf{Projeto Vindex}\\
Sistema de Detecção de Acidentes e Alerta Automático para Motociclistas

\vspace{0.5cm}

Igor Teixeira Lacerda\\
Lorenzo Bighetti Zauith\\
Rodrigo Costa de Araújo

\vspace{0.5cm}

Escola Politécnica da USP\\
Dezembro de 2025
\end{center}
\end{frame}

\end{document}
